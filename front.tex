\newcommand{\HRule}{\rule{\linewidth}{0.5mm}} 

\begin{titlepage}
\center
%	HEADING & LOGO
\textsc{
\Huge{Universidad de Sonora}\\[.5cm]
\Large
Departamento de Investigaci\'on en F\'{\i}sica\\[1cm] 
\includegraphics[width=3cm]{unison}\\[3cm]
Anteproyecto de Tesis de Maestria \\[.7cm] 
Proponente: Francisco Mart\'{\i}nez S\'anchez}\\[.7cm] 

%	TITLE 
\sffamily
\HRule \\[0.4cm]
\textbf{\LARGE Estudios de Simulaci\'on en la b\'usqueda de nuevos bosones ligeros durante la fase de alta luminosidad del experimento CMS del CERN}\\[0.2cm] 
\HRule \\[3cm]
 
%	AUTHORS & SUPERVISOR
\large
\begin{minipage}[t]{.6\textwidth}
\begin{flushleft}
Miembros del comit\'e:
\\
Alfredo Mart\'{\i}n Casta\~neda Hernandez (Director)\\
Susana Alvarez Garcia (tutor)\\
Marcelino Barbosa Flores (tutor)\\
\end{flushleft}

\end{minipage}\hfill
%\begin{minipage}[t]{.4\textwidth}

%\begin{flushright}
%\emph{Course Instructor} 

%Sandipan Bandyopadhyay\\ 

%\end{flushright}
%\end{minipage}
%\\[2cm]

%	DATE
{\today}\\[3cm]

\end{titlepage}


\begin{abstract}
El Modelo Est\'andar es la teor\'{\i}a cu\'antica de campo que hasta el momento de hoy describe de manera precisa las interacciones entre las part\'{\i}culas fundamentales y los diferentes tipos de fuerzas que experimentan las mismas. En el 2012 las colaboraciones experimentales ATLAS y CMS reportaron el descubrimiento de una nueva part\'{\i}cula cuyas propiedades son consistentes con el boson de Higgs, el cual seg\'un la teor\'{\i}a es el portador del llamado campo de Higgs, mecanismo por el cual las part\'{\i}culas adquieren masa, actualmente las propiedades de dicha part\'{\i}culas son estudiadas.  Sin embargo, dicho modelo falla en dar explicaci\'on a varios fen\'omenos como por ejemplo la materia oscura, cuya existencia se infiere por la interacci\'on con la materia visible, a pesar de los diferentes estudios a\'un se desconoce la composici\'on de dicha materia. El presente proyecto explora varios modelos te\'oricos que predicen la creaci\'on de part\'{\i}culas de materia oscura como producto de las colisiones de protones que viajan a velocidades relativistas como las producidos en el Gran Colisionador de Hadrones del CERN. Dichos modelos son estudiados por medio de simulaci\'on de monte carlo, donde se explora las diferentes propiedades del modelo y se simula la respuesta del detector, dicho estudio en el contexto del experimento CMS del CERN.  
\end{abstract}
