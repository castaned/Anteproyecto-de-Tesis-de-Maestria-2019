\newcommand{\HRule}{\rule{\linewidth}{0.5mm}} 

\begin{titlepage}
\center
%	HEADING & LOGO
\textsc{
\Huge{Universidad de Sonora}\\[.5cm]
\Large
Departamento de Investigacion en Física\\[1cm] 
\includegraphics[width=2cm]{unison}\\[3cm]
Anteproyecto de Tesis de Maestria}\\[.7cm] 

%	TITLE 
\sffamily
\HRule \\[0.4cm]
\textbf{\LARGE Estudios de Simulacion en la busqueda de nuevos bosones ligeros durante la fase de alta luminosidad del experimento CMS del CERN}\\[0.2cm] 
\HRule \\[3cm]
 
%	AUTHORS & SUPERVISOR
\large
\begin{minipage}[t]{.6\textwidth}
\begin{flushleft}
\emph{Miembros del comite:}
\\
Alfredo Martin Castaneda Hernandez (Director)\\
Susana Alvarez Garcia (tutor)\\
Marcelino Barbosa Flores (tutor)\\
\end{flushleft}

\end{minipage}\hfill
%\begin{minipage}[t]{.4\textwidth}

%\begin{flushright}
%\emph{Course Instructor} 

%Sandipan Bandyopadhyay\\ 

%\end{flushright}
%\end{minipage}
%`\\[2cm]

%	DATE
{\today}\\[3cm]

\end{titlepage}


\begin{abstract}
El Modelo Estandar es la teoria cuantica de campo que hasta el momento de hoy describe de manera precisa las interacciones entre las particulas fundamentales y los diferentes tipos de fuerzas que experimentan las mismas. En el 2012 las colaboraciones experimentales ATLAS y CMS reportaron el descubrimiento de una nueva particula cuyas propiedades son consistentes con el boson de Higgs, el cual segun la teoria es el portador del llamado campo de Higgs, mecanismo por el cual las particulas adquieren masa, actualmente las propiedades de dicha partiuclas son estudiadas.  Sin embargo, dicho modelo falla en dar explicacion a varios fenomenos como por ejemplo la materia oscura, cuya existencia se infiere por la interaccion con la materia visible, a pesar de los diferentes estudios aun se desconoce la composicion de dicha materia. El presente proyecto explora varios modelos teoricos que predicen la creacion de particulas de materia oscura como producto de las colisiones de protones que viajan a velocidades relativistas como las producidos en el Gran Colisionador de Hadrones del CERN. Dichos modelos son estudiados por medio de simulacion de monte carlo, donde se explora las diferentes propiedades del modelo y se simula la respuesta del detector, dicho estudio en el contexto del experimento CMS del CERN.  
\end{abstract}
