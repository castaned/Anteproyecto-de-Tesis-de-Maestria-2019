\newcommand{\HRule}{\rule{\linewidth}{0.5mm}} 

\begin{titlepage}
\center
%	HEADING & LOGO
\textsc{
\Huge{Universidad de Sonora}\\[.5cm]
\Large
Departamento de Investigacion en Física\\[1cm] 
\includegraphics[width=2cm]{unison}\\[3cm]
Anteproyecto de Tesis de Maestria}\\[.7cm] 

%	TITLE 
\sffamily
\HRule \\[0.4cm]
\textbf{\LARGE Estudios de Simulacion en la busqueda de nuevos bosones ligeros durante la fase de alta luminosidad del experimento CMS del CERN}\\[0.2cm] 
\HRule \\[3cm]
 
%	AUTHORS & SUPERVISOR
\large
\begin{minipage}[t]{.6\textwidth}
\begin{flushleft}
\emph{Miembros del comite:}
\\
Alfredo Martin Castaneda Hernandez (Director)\\
Susana Alvarez Garcia (tutor)\\
Marcelino Barbosa Flores (tutor)\\
\end{flushleft}

\end{minipage}\hfill
%\begin{minipage}[t]{.4\textwidth}

%\begin{flushright}
%\emph{Course Instructor} 

%Sandipan Bandyopadhyay\\ 

%\end{flushright}
%\end{minipage}
\\[2cm]

%	DATE
{\today}\\[3cm]

\end{titlepage}


\begin{abstract}
El modelo estandar es el marco teorico que describe las interacciones y fuerzas entre las particulas fundamentales, uno de sus mayores logros es el de postular el mecanismo por el cual las particulas adquieren masa (campo de Higgs), dicho mecanismo viene acompaniado por un portador, el boson de Higgs, el cual fue reportada su observacion por las colaboraciones experimentales ATLAS y CMS del CERN en verano del 2012. Sin embargo, el modelo estandar falla en describir fenomenos de la naturaleza como lo son la materia y energia oscura entre otros. El presente proyecto explora varios modelos teoricos que predicen la creacion de particulas de materia oscura en las colisiones del Gran Colisionador de Hadrones del CERN. 
\end{abstract}
